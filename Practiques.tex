\documentclass[a4paper,addpoints,12pt,answers]{exam}
%,answers
\usepackage[utf8]{inputenc}
\usepackage{amssymb,amsmath,amsfonts,amsthm}
%%%%%%%%%%%%%%%%%%%%%%%%%%%%%%
\usepackage{graphicx}
\usepackage[usenames]{color}
\usepackage{hyperref}
\usepackage{rotating}
\usepackage{fancybox}
%\usepackage{calendar}
\usepackage{amsfonts}
\usepackage{amsmath}
\usepackage{amssymb}
\usepackage[spanish]{babel}
\usepackage{multicol}
\usepackage[table]{xcolor}

%\usepackage{ceu}
\def\barraceu#1{
\noindent
\mbox{$
\color{celesteceu}\rule{\textwidth}{1.1#1}
\hspace{-\textwidth}
\color{verdeceu}\rule{\textwidth}{0.95#1}
\hspace{-\textwidth}
\color{rojoceu}\rule{\textwidth}{0.8#1}
\hspace{-\textwidth}
\color{azulceu}\rule{\textwidth}{0.55#1}
$
}
\color{black}
}

\oddsidemargin 1.5cm
\voffset -1in
\hoffset -1in
\topmargin 0cm
\textwidth 18.50truecm
\textheight 22.50truecm
\headheight 4cm


\def\labelenumi{\theenumi \textbf{)}}
\def\theenumi{\textbf{\alph{enumi}}}
\def\Pr{\textrm{Pr}}

\def\cabeza{\makebox[\textwidth]{
\includegraphics[width=0.3\textwidth]{./ceu_bis.pdf} \hfill \begin{tabular}[b]{r}
Physique\\
Dégrée en Kine
\\
Décalogue pour les pratiques de physique
\\
Elche, 2021
\end{tabular}
}
}

\def\cabezaconnombre{
\cabeza
\\
\bigskip
}

%opening

\title{Física}
\author{Problemas}
\date{}
\def\Pr{\mathop{\mathrm{Pr}}}
\pagestyle{headandfoot}
\begin{document}
\firstpageheader{}{\cabezaconnombre}{}
\runningheader{}{\cabeza \\ Página~\thepage~of 3}{}
\runningfooter{}{}{}
% \def\instructions{
% \textbf{Intructions}(Read this before starting):
% \begin{enumerate}
% \item Answer the questions in the spaces provided on the
% question sheets. If you run out of room for an answer,
% continue on the back of the page or on the final page.
% \item The duration of the exam is 1 hour and 45 minutes.
% \item Any kind of communication with someone inside or outside the
%   room is strictly forbidden whatever the subject. It would be
%   considered as cheating and the student would be expelled from the
%   roon and the exam would not be marked. \textbf{Cell phone should be
%     switched off} during the exam.
% \item Every answer should be explained. Please, write down any calculation
%   you have typed on your computer in order to get a number. Answers
%   based on an alone number is not to be considered.   
% \item  This exam has \numquestions\ questions, for a total of
%   \numpoints\ points.
% \item The first part has 5 questions, for a total of 50 points.
%   All the students have to do it. 
% \item The question 6 or 7, for a total of 50 points and it should be answered
% by the students that have not followed continuous evaluation. 
% \end{enumerate}
% }

% %\maketitle

% \instructions

% \newpage

\textbf{Décalogue pour les pratiques de physique:}

\bigskip

L’objective dans les pratiques de physique est de développer les compétences liées \textbf{(a) à l'emploi de la méthode scientifique dans la recherche de réponses aux questions liées à la physique, (b) au travail d'équipe et (c) à la rédaction de courts textes scientifiques}. Pour atteindre ces objectifs:

\bigskip


\begin{enumerate}
\item L’assistance est obligatoire.
\item Chaque groupe de pratiques sera divisé en équipes, composées de trois personnes maximum. Il est recommandé, bien que non obligatoire, que chaque équipe soit la même pour chacune des pratiques prévues.
\item L'objectif de la pratique sera de répondre à une question de nature scientifique dont l'énoncé sera remis par écrit aux étudiants.
\item Chaque équipe élira un porte-parole qui représentera l'équipe.
\item Chaque équipe travaillera de manière indépendante à la recherche de la réponse, en utilisant exclusivement les matériaux et les données indiqués par écrit.
\item Chaque équipe pourra collaborer et partager des informations avec le reste des équipes participant à la pratique.
\item Pour conclure et valider la pratique, il sera nécessaire de soumettre un rapport écrit, avec le nom et prénom de chacun des membres de l'équipe, où il sera détaillé:
\begin{enumerate}
\item[(i)] Matériel et méthode utilisés pour répondre à la question proposé dans la pratique.
\item[(ii)] La stratégie utilisée pour répondre à la question proposé dans la pratique.
\item[(iii)] Petite explication des calculs nécessaires pour la mise en œuvre de l'objectif de la pratique.
\item[(iv)] Conclusion de l'expérience et remarques finales.
\end{enumerate}  
\item L'équipe aura validé la pratique une fois le professeur donne le OK pour télécharger à l'intranet le rapport écrit.
\item Chaque équipe doit télécharger à l'intranet un rapport indépendant des autres équipes participant dans la pratique.
\end{enumerate}
\end{document}
